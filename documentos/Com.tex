\subsection{Comparación entre Compiladores}
Un compilador no es mas que un programa que basicamente toma el codigo C
y lo traduce a codigo de maquina. Existen varios para C, tales como como GCC (el cual se usa en gran parte de este trabajo), Clang y el Compilador de Intel.
Lo que hicimos en este experimento fue compilar los filtros con GCC y Clang en sus versiones 5.3.1 y 3.8.0 respectivamente, para luego realizar la comparación.
%Lo que vamos a hacer es compilar el codigo con Clang/GCC y ver si encontramos diferencias.Para comparar, se van a usar la version 4.8.4 de GCC y la 3.4 de Clang.
%Para realizar la comparacion, vamos a compilar el TP con GCC y luego con Clang.
%Para empezar, vamos a comparar con optimizacion nula
Nuestra hipótesis para este experimento es que el compilador de GCC optimice más el código producido que Clang, ya que en caso contrario, este último estaría instalado por defecto
en muchos de los sistemas operativos existentes en lugar de GCC (teniendo en cuenta las muchas otras razones por las que puede no ser el caso).


Aqui el grafico de GCC

\begin{figure}[H]
    \centering
    \begin{floatrow}
      \ffigbox[\FBwidth]{\caption{Filtros con GCC -O0}}{%
        \includegraphics[scale=.55]{./imagenes/GCC0.jpg}
      }
    \end{floatrow}
\end{figure}


Aca vemos que Cropflip, es el filtro mas rapido. Es esperable ya que solo hace accesos a memoria mientras que sepia y cropflip hace operaciones arimeticas

Aqui el grafico de Clang

\begin{figure}[H]
    \centering
    \begin{floatrow}
      \ffigbox[\FBwidth]{\caption{Filtros con Clang -O0}}{%
        \includegraphics[scale=.55]{./imagenes/Clang0.jpg}
      }
    \end{floatrow}
\end{figure}

Vemos que en Cropflip gana Clang, pero en Sepia y LDR gana GCC, en sepia es mas notorio mientras que en LDR la diferencia es bastante menor.


Ahora, vamos a repetir el mismo experimento pero variando las flags, esta vez vamos a usar 03 y como hipotesis, esperamos ver un comportamiento similar al visto en O0.
\begin{figure}[H]
    \centering
    \begin{floatrow}
      \ffigbox[\FBwidth]{\caption{Filtros con GCC -O0}}{%
        \includegraphics[scale=.55]{./imagenes/GCC3.jpg}
      }
    \end{floatrow}
\end{figure}


\begin{figure}[H]
    \centering
    \begin{floatrow}
      \ffigbox[\FBwidth]{\caption{Filtros con Clang -O0}}{%
        \includegraphics[scale=.55]{./imagenes/Clang3.jpg}
      }
    \end{floatrow}
\end{figure}

Efectivamente, se da el comportamiento previsto, en Cropflip gana Clang mientras que en los otros gana GCC
